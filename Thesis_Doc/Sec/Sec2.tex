\documentclass[../thesis_main.tex]{subfiles}

\begin{document}
\doublespacing

This paper is grounded in prior work in both education and in economics. In education, there have been several studies analyzing grades as a student outcome; both their predictive ability and reliability, as well as their potential shortcomings. As cited above, particularly relevant to this paper are several papers from the Chicago Consortium on School Research examining grades in the context of Chicago Public Schools. \citet{eastonPredictivePowerNinthGrade2017}, for example, find that freshman year GPA is a powerful predictor of student outcomes, much more powerful, in fact, than test scores. This paper also acknowledges, however, that the reason grades so strongly signal later success is not explained, and hypothesizes that they may be measuring other unobserved components of student achievement such as effort, behavior, attendance, or attitude. As a follow up to this study, \citet{allensworthWhyStudentsGet2018} attempt to analyze the components which contribute to the grades students' receive. While they confirm that including factors such as attendance in a model of grades greatly reduces the amount of between-teacher variation in grading effect, they also acknowledge that there is a significant component of this variation which remains unaccounted for. 

There are two main obstacles to linking this remaining between-teacher variation in grades directly to idiosyncratic teacher grade effects. First, students may be non-randomly sorted to teachers on unobservable characteristics, leading to a potential selection bias in the estimation of these grade effects. This issue is widely addressed in the economics literature on teacher value added models (VAM), such as \citet{chettyMeasuringImpactsTeachers2014}, \citet{kaneEstimatingTeacherImpacts2008}, or \citet{jacksonTeacherEffectsTeacherRelated2014}. These papers exploit the fact that teachers see different cohorts of students each year to estimate the actual educational impact a teacher has on their students. While this is in some ways the opposite of attempting to estimate the non-educational aspects of grades, it suffers from the same risk of potential non-random assignment of students to teachers. This literature therefore includes a wide-variety of tests for exactly this assumption, which I attempt to adapt to the particular setting of CPS. Outside of the value added literature, I also drew from \citet{abdulkadirogluChartersLotteriesTesting2016} and \citet{kolesarIdentificationInferenceMany2015} to inform the methodological approach and assumptions used in my identification argument. 

The other main obstacle in identifying the residual between-teacher variation in grades with individual teacher grade effects is the inherent difficulty in disentangling the grade a student receives from their actual performance in the class. Even if a given teacher's students all, on average, receive higher grades with that teacher than they do in any other class, it is not clear if this indicates that this teacher is an "easy-A" or if they are inherenlty better at communicating information to their students than any other teacher.

%% THIS PARAGRAPH NEEDS LOTS OF WORK
The primary innovation of this paper relative to the literature is the methodology I employ to attempt to address this issue. Under the functional assumption discussed above, using a grade outcome differenced across multiple years removes the component of teacher grading correlated with persistent gains in human capital. This framework closely follows \citet{gilraineMakingTeachingLast2020}, which identifies both a transitory ``short-run'' and a permanent ``long-run'' component of traditional teacher value-add. The model used to separate these two effect, when applied to the setting of grades rather than test scores, allows us to in turn attempt to idenitfy the componenet of a teacher's grading which is transitory and unobserved in future years. While this approach may be unable to precisely explain the unobservables which are contributing to this observed grade effect, even examining the proportion of the grade effect which is separate from human capital accumulation represents a new pathway by which future resarch can attempt to determine how grading is related to student outcomes. 


The structure of the paper is as follows. In Section \ref{section:framework}, I present the theoretical and framework for the primary analysiss. In Section \ref{section:data}, I detail both the data used for the analysis and the construction of the analytic sample, and the main statistical procedure for estimating the teacher-level grade effects. In Section\ref{section:results}, I detail the results of the primary analyses, and in Section \ref{section:tests}, I highlight the results of the empirical tests on the assumptions. Finally, Section VII concludes the paper. 


\end{document}