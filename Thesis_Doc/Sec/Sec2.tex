\documentclass[../thesis_main.tex]{subfiles}

\begin{document}
\doublespacing
\section{Relevant Literature}
\label{section:lit}

This paper is well-grounded in prior work in both education and in economics. In education, there have been several studies analyzing grades as a student outcomes; both their predictive ability and reliability, as well as their potential shortcomings. As cited above, particularly relevant to this thesis are several papers from the Chicago Consortium on School Research examining grades in the context of Chicago Public Schools. \cite{eastonPredictivePowerNinthGrade2017}, for example, find that freshman year GPA is a powerful predictor of student outcomes, much more powerful, in fact, than test scores. This paper also acknowledges, however, that the reason grades so strongly signal later success is not explained, and hypothesizes that they may be measuring other unobserved components of student achievement such as effort, behavior, attendance, or attitude. As a follow up to this study, \cite{allensworthWhyStudentsGet2018} attempt to analyze the components which contribute to the grades students' receive. While they confirm that including factors such as attendance in a model of grades greatly reduces the amount of between-teacher variation in grading effect, they also acknowledge that there is a significant component of this variation which remains unaccounted for. This paper attempts to add to this literature by further quantifying that effect and examining how it affects student outcomes. While we are still unable to precisely explain the unobservables which may be contributing to this, the analysis in this paper provides a better picture of how this remaining variation functions in practice, laying the groundwork for further research to examine the pathways by which grading is related to student outcomes. 

In economics, this paper is rooted well within the empirical literature on human capital accumulation, and, specifically, research on teacher value added models (VAM), such as \cite{chettyMeasuringImpactsTeachers2014}, \cite{kaneEstimatingTeacherImpacts2008}, or \cite{jacksonTeacherEffectsTeacherRelated2014}. These papers exploit the fact that teachers see different cohorts of students each year to estimate the actual educational impact a teacher has on their students. While this is in some ways the opposite of attempting to estimate the non-educational aspects of grades, this literature still provides a theoretical and empirical framework for examining grading patterns and the relationships between teachers and students which I here incorporate into my models. The specific framework I employ is most closely related to \cite{gilraineMakingTeachingLast2020}, which identifies both a transitory ``short-run'' and a permanent ``long-run'' component of traditional teacher value-add. This paper adapts the identification argument required to separate these two effects to identify instead the extent to which students' grades measure the transitory effect of idiosyncratic teacher grading versus the more long-term impact of actual gains in student achievement. Empirically, this paper strives to use a wide array of rigorous econometric methods both to identify teacher grading impact and to verify our assumptions in the data; to this extent, I  build off prior empirical work by \cite{abdulkadirogluChartersLotteriesTesting2016} and \cite{kolesarIdentificationInferenceMany2015}.

\end{document}